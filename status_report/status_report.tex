    
\documentclass[11pt]{article}
\usepackage{times}
    \usepackage{fullpage}
    
    \title{Privacy Preservation in Human Motion Analysis}
    \author{Boondirek Kanjanapongporn - 2833456k}

    \begin{document}
    \maketitle
    
    
     

\section{Status report}

\subsection{Proposal}\label{proposal}

\subsubsection{Motivation}\label{motivation}

Human motion analysis is a powerful tool to extract biomarkers for disease progression in neurological conditions, such as Parkinson's disease and Alzheimer’s. In healthy individuals, human motion analysis has revealed indices that relate to both physical and mental fatigue. Current progress in machine learning has made massive steps towards intelligent systems that match or even surpass human abilities in applications using computer vision. However, the translation of these methodologies in the healthcare domain is far more challenging due to several ethical issues, safety, and privacy concerns that need to be addressed and resolved before use in scale.

\subsubsection{Aims}\label{aims}

In this project, we are going to explore the ability to identify people from radar signatures data set. Then we will explore ways to preserve the privacy of the data by identifying latent factors and generative sources that reflect bio metrics and filter them out from the processing pipeline.

\subsection{Progress}\label{progress}

\begin{itemize}
  \item Done background research on radar technologies on human motion
  \item Downloaded radar signatures data set from College of Science and Engineering, University of Glasgow
\end{itemize}

\subsection{Problems and risks}\label{problems-and-risks}

\subsubsection{Problems}\label{problems}

\begin{itemize}
  \item Learning curve of Machine Learning and Deep Learning
\end{itemize}

\subsubsection{Risks}\label{risks}

\begin{itemize}
  \item Study Machine Learning and Deep Learning from online courses
\end{itemize}

\subsection{Plan}\label{plan}
\textbf{Semester 1}
\begin{itemize}
  \item Week 1-3: Study Machine Learning and Deep Learning
  \item Week 4-5: Understand the example source code to modify for my project
  \item Week 6-7: Follow matlab script and replicate using numpy
  \item Week 8-9: Explore the ability to identify people from radar signatures
  \item Week 10-11: Explore ways to extract bio metrics in order to filter them out
  \item Week 12: Explore ways to extract bio metrics in order to filter them out
\end{itemize}

\subsection{Ethics and data}\label{ethics}
\emph{Specify what ethical approval you need to do your evaluation and how you are approaching it. This is mandatory. 
Specify what data you expect to collect in your evaluation. Explain how this data will help you evaluate your project.
}

Options for ethics:

\item This project does not involve human subjects or data. No approval required.
\item I have verified that the ethics checklist will apply to any evaluation I need to do. I will sign and complete the checklist.
\item I have sought ethical guidance from the School's ethics convener and I will:
\begin{itemize}
    \item Proceed under specific instructions from the Ethics convener (e.g. modified checklist).
    \item Apply for College Ethics Board approval.
    \item Other procedure (give details)
\end{itemize}   

\end{document}

